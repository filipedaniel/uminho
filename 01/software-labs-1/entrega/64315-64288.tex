\documentclass[a4paper]{article}
\usepackage{a4}
\usepackage{moreverb}
\usepackage{palatino}
\usepackage[utf8x]{inputenc}
\usepackage[portuges]{babel}
\newcommand{\OlxUm}{{O\hskip-.05ex\raisebox{-.4em}L\hskip-0.8ex X\raisebox{.4em}{{\footnotesize\sc um}}}}
\usepackage{graphicx}
\DeclareGraphicsExtensions{.jpg}
\graphicspath{{./imagens/}}
\usepackage{verbatim}
%----------------------------------------------------------------------------
\title{Relatório do Projecto de Laboratórios de Informática I (2012/2013)}

\author{
\\
	64315 - Filipe Ribeiro
\\
	64288 - Nelson Gomes
}

\date{Jan.\ 2012}
%------------------------------------------------------------------
\begin{document}
\begin{figure}
\begin{center}
\includegraphics[]{UM.jpg}
\end{center}
\end{figure}
\maketitle
\newpage
%------------------------------------------------------------------
%xxxxxxxxxxxxxxxxxxxxxxxxxxxxxxxxxxxxxxxxxxxxxxxxxxxxxxxxxxxxxxxxxxxxxxxxxxxxxxxxxxxxxxxxxxxxxxxxxxxxxxxxxxxxxxxxxxxxxxxxxxxxxxxxxxxxxxxxxxx
\begin{abstract}
Este relatório descreve as varias etapas pedidas no âmbito deste mesmo projeto. Serão demonstradas e explicadas oito das nove pedidas. Cada etapa esta dividida em varias partes, o que e pertendido que a função faça, o tipo, funções auxiliares, código Haskell e a análise ao cógido.
\end{abstract}
%------------------------------------------------------------------
\section{Programação}
\begin{enumerate}
\item 
Implementar a função $idsOk :: House -> Bool$ que verificar que não há identificadores repetidos (nas duas listas).
\begin{itemize}
\item {Tipo de dados:}
\begin{verbatim}
      idsOk :: House -> Bool
\end{verbatim}
\paragraph{}O objetivo e ao receber um $House$ teste ($Bool$) se de todos os identificadores existem repetidos.
\item {Codigo Haskell:}
\begin{verbatim}
idsOk :: House -> Bool
idsOk h = let l = hrunning h ++ hfinished h
              m = nub(map actid l)
          in  length m == length l
\end{verbatim}
\item {Análise do código:}
\paragraph{}O $let$ cria duas variáveis, $l$ e $m$, a primeira cria uma lista com os artigos que estão no $hrunning$ $h$, e no $hfinished$ $h$. A segunda, utiliza duas funções predefinidas do Haskell $nub$\footnote{nub :: $Eq$ $a => [a] -> [a]$} e $map$\footnote{map :: $(a -> b) -> [a] -> [b]$}.
\begin{verbatim}
      ... (map actid l)
\end{verbatim}
\paragraph{}O $map$ cria uma lista de todos os $actid$ pertencentes há lista $l$, que se construiu antes. 
\begin{verbatim}
      ... m = nub(map actid l)
\end{verbatim}
\paragraph{}O $nub$ tem um papel importante, ao receber a lista vinda do $map$, constroi outa lista com os números pertencentes há lista que recebe, e se existir repetidos exclui esses mesmos. Logo a comparação final, 
\begin{verbatim}
      ... in  length m == length l
\end{verbatim}
como mostra no código, se o comprimento de $l$ (lista original), for igual ao comprimento da lista $m$ (não existir identificadores repetidos) a função toma o valor lógico $True$ e se não for o caso, $False$.
\end{itemize}

%xxxxxxxxxxxxxxxxxxxxxxxxxxxxxxxxxxxxxxxxxxxxxxxxxxxxxxxxxxxxxxxxxxxxxxxxxxxxxxxxxxxxxxxxxxxxxxxxxxxxxxxxxxxxxxxxxxxxxxxxxxxxxxxxxxxxxxxxxxxx
%xxxxxxxxxxxxxxxxxxxxxxxxxxxxxxxxxxxxxxxxxxxxxxxxxxxxxxxxxxxxxxxxxxxxxxxxxxxxxxxxxxxxxxxxxxxxxxxxxxxxxxxxxxxxxxxxxxxxxxxxxxxxxxxxxxxxxxxxxxxx

\newpage
\item
Acrescentar a funcão $idKey :: House -> Bool$ que verifica se o identificador $actid$ de um dado item o identifica univocamente.
\begin{itemize}
\item {Tipo de dados:}
\begin{verbatim}
      idKey :: House -> Bool
\end{verbatim}
\paragraph{}O objetivo é, ao receber um $House$ teste ($Bool$) se o identificador de um dado item o identifica univocamente.

\item {Código Haskell:}
\begin{verbatim}
idKey :: House -> Bool
idKey h = let l = hrunning h ++ hfinished h
              m = idsOk h      
          in if (m == False) then verifica l else True

verifica :: [Auction] -> Bool
verifica [] = True
verifica (h:t) | ajuda h (h:t) = verifica t
               | otherwise = False

ajuda :: Auction -> [Auction] -> Bool
ajuda _ [x] = True
ajuda a t = let b = nub(filter((==actid a).actid) t)
            in length b == 1
\end{verbatim}
\item {Análise do código:}
\paragraph{}Nesta função também o $let$ cria duas variaveis $l$ e $m$, o $l$ é novamente uma lista com os artigos que estão no $hrunning$ $h$, e os que estão no $hfinished$ $h$, mas $m$ chama a função anterior, $idsOk$, visto que se $idsOk$ toma o valor lógico de $True$, que diz que não ha identificadores repetidos, a função que estamos a analisar toma também o valor de $True$ e termina.
\paragraph{}Caso a função $idsOk$ toma o valor lógico $False$ a função em estudo chama uma outra função, uma função auxiliar: 
\begin {verbatim}
verifica :: [Auction] -> Bool
verifica [] = True
verifica (h:t) | ajuda h (h:t) = verifica t
               | otherwise = False
\end {verbatim}
\paragraph{} Esta função ao receber uma lista de $Auction$ vai comparar recursivamente o identificador de cada item com o do resto dos itens e verifica se existe outro igual, para isso recorre a outra função:
\begin {verbatim}
ajuda :: Auction -> [Auction] -> Bool
ajuda _ [x] = True
ajuda a t = let b = nub(filter((==actid a).actid) t)
            in length b == 1
\end {verbatim}
\paragraph{}Esta função auxiliar tém um papel simples mas importante, dado um $Auction$ e $[Auction]$ testa se existe algum identificador igual ao do $Auction$ na $[Auction]$, na sua execução cria uma variavel, $b$, que utiliza duas funções já existentes no Haskell, $nub$ e $filter$\footnote{filter :: $(a -> Bool) -> [a] -> [a]$} o resultado da aplicação desta ultima função, 
\begin {verbatim}
       ... (filter((==actid a).actid) t)     
\end {verbatim}
cria uma lista, que indo aos $actid$ de $t$, seleciona os que são iguais ao $actid$ de $a$, depois com o $nub$, se o comprimento da lista dada for um, significa que so existe um, o proprio $actid$ $a$, podendo-se deduzir que de facto não existe identificadores repetidos, tomando a função $ajuda$ o valor de $True$.
\paragraph{}Voltando há função $verifica$, se o resultado da função $ajuda$ for $True$ volta a fazer o mesmo procedimento mas desta vez aplicado só há cauda da lista.
\begin {verbatim}
       ... | ajuda h (h:t) = verifica t
\end {verbatim}
\paragraph{}Caso contrário o resultado da função $verifica$ é $False$, o que quer dizer que existe um identificador de um dado item que não o identifica univocamente, tomando também a função principal o valor de $False$.
\end{itemize}

%xxxxxxxxxxxxxxxxxxxxxxxxxxxxxxxxxxxxxxxxxxxxxxxxxxxxxxxxxxxxxxxxxxxxxxxxxxxxxxxxxxxxxxxxxxxxxxxxxxxxxxxxxxxxxxxxxxxxxxxxxxxxxxxxxxxxxxxxxxxx
%xxxxxxxxxxxxxxxxxxxxxxxxxxxxxxxxxxxxxxxxxxxxxxxxxxxxxxxxxxxxxxxxxxxxxxxxxxxxxxxxxxxxxxxxxxxxxxxxxxxxxxxxxxxxxxxxxxxxxxxxxxxxxxxxxxxxxxxxxxxx

\item
Será que $idKey$ implica $idsOk$? Ou será que $idsOk$ é que implica $idKey$?
      Ou nenhuma dessas implicações se verifica? Justifique a sua resposta.
\paragraph{}Na nossa opinião $idKey$ implica $idsOk$ pelo facto de, sabendo que $idsOk$ toma o valor lógico $Tru$e então $idKey$ toma de imediato o mesmo valor de $True$ e sendo o contrario, havendo $actid$ repetidos é que $idKey$ faz a sua verificação.

%xxxxxxxxxxxxxxxxxxxxxxxxxxxxxxxxxxxxxxxxxxxxxxxxxxxxxxxxxxxxxxxxxxxxxxxxxxxxxxxxxxxxxxxxxxxxxxxxxxxxxxxxxxxxxxxxxxxxxxxxxxxxxxxxxxxxxxxxxxxx
%xxxxxxxxxxxxxxxxxxxxxxxxxxxxxxxxxxxxxxxxxxxxxxxxxxxxxxxxxxxxxxxxxxxxxxxxxxxxxxxxxxxxxxxxxxxxxxxxxxxxxxxxxxxxxxxxxxxxxxxxxxxxxxxxxxxxxxxxxxxx

\item
Acrescentar o predicado $allIds :: House -> Bool$ que verifica se a alocação de identificadores a leilões é sequencial.
\begin{itemize}
\item {Tipo de dados:}
\begin{verbatim}
      allIds :: House -> Bool
\end{verbatim}
\paragraph{}O objetivo e ao receber um $House$, teste ($Bool$) se os identificadorres de um leilão estao de forma sequencial, quer seja crescente quer decrescente.
\newpage
\item {Código Haskell:}
\begin{verbatim}
allIds :: House -> Bool
allIds h = 
  let (a,b)=((map actid (hrunning h)),(map actid (hfinished h)))
  in (abOk a b)

abOk :: [Int] -> [Int] -> Bool
abOk x y = let m = f x
               n = g x
               m1 = f y
               n1 = g y
           in (m || n) && (m1 || n1)

f :: [Int] -> Bool 
f [] = True
f [x] = True 
f (h:x:xs) = (h>=x)&&(f (x:xs))
g :: [Int] -> Bool
g [] = True
g [x] = True 
g (h:x:xs) = (h<=x)&&(g (x:xs))
\end{verbatim}

\item {Análise do código:}
\paragraph{}A função principal,
\begin{verbatim}
allIds :: House -> Bool
allIds h = 
  let (a,b)=((map actid (hrunning h)),(map actid (hfinished h)))
  in (abOk a b)
\end{verbatim}
constroi duas variáveis, crindo o par, $(a,b)$, a primeira, usando o $map$ cria uma lista com todos os $actid$ pertencentes ao $hrunning$ $h$, enquanto a segunda, cria também uma lista de $actid$, mas desta vez pertencentes ao $hfinished$ $h$.
\paragraph{}Depois de criada as variáveis, a função chama uma função auxiliar $abOk$,
\begin{verbatim}
abOk :: [Int] -> [Int] -> Bool
abOk x y = let m = f x
               n = g x
               m1 = f y
               n1 = g y
           in (m || n) && (m1 || n1)
\end{verbatim}
esta função, verifica se tanto os $actid$ de $a$ e $b$ estão de alguma forma ordenados. Em comparação com a função principal, esta também utiliza outras funções auxiliares: \\
\newline
\begin{tabular}{l l}
$f :: [Int] -> Bool$ & $g :: [Int] -> Bool$\\
$f [] = True$ & $g [] = True$\\
$f [x] = True$ & $g [x] = True$\\ 
$f (h:x:xs) = (h>=x)\&\&(f (x:xs))$ & $g (h:x:xs) = (h<=x)\&\&(g (x:xs))$\\
\end{tabular}
\newpage
\paragraph{}Estas duas funções tem um papel minucioso, a função $f$ verifica se a lista que recebe é decrescente e a $g$ se é crescente. Guardando os valores dos testes de cada uma das listas em quatro variaveis, $m$, $n$, $m1$, $n1$. 
\paragraph{}Voltanto há função principal, a operação final, 
\begin{verbatim}
      ... (m || n) && (m1 || n1)
\end{verbatim}
sendo que as listas só precisão de estar ordenadas de uma das formas, dai a condição $(||)$, mas independentemente da ordenação, precisão de estar ordenadas, dai a condição $(\&\&)$.\\
\end{itemize}

%xxxxxxxxxxxxxxxxxxxxxxxxxxxxxxxxxxxxxxxxxxxxxxxxxxxxxxxxxxxxxxxxxxxxxxxxxxxxxxxxxxxxxxxxxxxxxxxxxxxxxxxxxxxxxxxxxxxxxxxxxxxxxxxxxxxxxxxxxxxx
%xxxxxxxxxxxxxxxxxxxxxxxxxxxxxxxxxxxxxxxxxxxxxxxxxxxxxxxxxxxxxxxxxxxxxxxxxxxxxxxxxxxxxxxxxxxxxxxxxxxxxxxxxxxxxxxxxxxxxxxxxxxxxxxxxxxxxxxxxxxx

\item
É fácil de ver, exercitando \OlxUm, que a função $auctionBid$ não está implementada convenientemente, pois:
\begin{itemize}
\item	após fazer \textsf{Reset}, se licitar o item 1 por 61 euros e depois por 1 (um) euro,
	verá que o sistema aceita qualquer oferta, mesmo que seja inferior à melhor oferta
	até ao momento, violando o princípio do \emph{quem dá
	mais ganha} inerente a qualquer leilão;
\item	se licitar um item que não existe, por exemplo o item 99, o sistema
	dá erro.
\end{itemize}
Reescreva a função por forma a nenhuma das anomalias identificadas acima se verificar.

\begin{itemize}
\item{Versão melhorada: Tipo de dados:}
\begin{verbatim}
     auctionBid :: House -> Int -> String -> Int -> House
\end{verbatim}
\paragraph{}O objectivo e aumentar o rigor do leilão, ao receber um $House$, o identificador do item, o nome de quem esta a tentar comprar e o valor que oferece pelo produto. 
\item{Versão melhorada: Codigo Haskell:}
\end{itemize}
\begin{verbatim}
auctionBid :: House -> Int -> String -> Int -> House
auctionBid h id bidder bid =  
 let curr = (filter ((==id).actid) (hrunning h))
 in if (curr == []) then error "Item não existente" else 
        if (podee curr bid) then (add h curr id bidder bid) 
              else error "Existe um valor de licitação superior"

podee :: [Auction] -> Int -> Bool
podee h i = let l = head(h)
            in i > (actvalue l) 

add :: House -> [Auction] -> Int -> String -> Int -> House
add h curr id bidder bid = 
  let r = filter ((/=id).actid) (hrunning h)
      cur = head(curr)
      newa = Auction (actid cur) (actowner cur) (actdesc cur) 
                     (actclass cur) (actvalinc cur) bid bidder
  in House (newa:r) (hfinished h)
\end{verbatim}
\newpage
\begin{itemize}
\item{Análise do código:}
\end{itemize}
\paragraph{}A grade diferença entre a versão antiga e a melhorada, é como foi referido, a rigor do leilão, dando avisos a quem esta a participar no leilão. Existem dois avisos possiveis,
\begin{verbatim}
      ... error "Item não existente" else 
            ... 
              ... error "Existe um valor de licitação superior"
\end{verbatim}
o primeiro, e quando o utilizador pertende adquirir um item, colocando um identificador inixestente, o segundo e quando o valor oferecido pelo item e inferior a um já oferecido.\\
\paragraph{}A primeira tarefa que a função $auctionBid$ faz e criar uma variavel, $curr$, usando o $filter$, cria uma lista de $Auction$, pertencentes ao $hrunning$ $h$, com os que tenham $actid$ iduais ao $id$ dado, e faz logo de seguida o teste, se a lista for vazia, que significa que não ha identificadores iguais, aparece o primeiro erro. Se não for esse o caso faz a verificação sobre o valor da licitação, chamando a função auxiliar:
\begin{verbatim}
      podee :: [Auction] -> Int -> Bool
      podee h i = let l = head(h)
                  in i > (actvalue l) 
\end{verbatim}
\paragraph{}Esta função testa se o valor oferecido e superior ao valor já anteriormente oferecido pelo item, a utilização do $head$, apesar de a lista $h$ ter normalmete um elemento, é para ter selecionado um elemento e não uma lista, para facilitar o teste seguinte.
\begin{verbatim}
      ... in i > (actvalue l) 
\end{verbatim} 
\paragraph{}Esta compara o valor a que o utilizador está a oferecer pelo item com o valor já existente de licitações anteriores, se for não for maior aparece o segundo erro.
\paragraph{}Se tiver passado por estes dois testes, é precisso registar o licitação, chamando assim outra função,
\begin{verbatim}
add :: House -> [Auction] -> Int -> String -> Int -> House
add h curr id bidder bid = 
 let r = filter ((/=id).actid) (hrunning h)
     cur = head(curr)
     newa = Auction (actid cur) (actowner cur) (actdesc cur) 
                    (actclass cur) (actvalinc cur) bid bidder
 in House (newa:r) (hfinished h)
\end{verbatim}
\newpage
\paragraph{}Esta função ao receber todos os dados de uma licitação, regista-os no $House$ que recebe. Cria a variavel $r$, que indo aos $actid$ dos $Auction$ que recebe da lista $hrunning$ $h$ cria uma lista dos $Auction$ cujo $actid$ sejam diferentes do $id$ dado na função. Cria também o $cur$ que é o item que o utilizador  esta a licitar, e cria um novo $Auction$ com os dados iguais ao do $cur$ juntando o valor licitado e o nome do comprador. Juntando no final toda a imformação no $House$, e junta a lista $hfinished$ $h$.

%xxxxxxxxxxxxxxxxxxxxxxxxxxxxxxxxxxxxxxxxxxxxxxxxxxxxxxxxxxxxxxxxxxxxxxxxxxxxxxxxxxxxxxxxxxxxxxxxxxxxxxxxxxxxxxxxxxxxxxxxxxxxxxxxxxxxxxxxxxxx
%xxxxxxxxxxxxxxxxxxxxxxxxxxxxxxxxxxxxxxxxxxxxxxxxxxxxxxxxxxxxxxxxxxxxxxxxxxxxxxxxxxxxxxxxxxxxxxxxxxxxxxxxxxxxxxxxxxxxxxxxxxxxxxxxxxxxxxxxxxxx

\item Supondo que o formato interno $House$ muda para
\begin{verbatim}
data NHouse = NHouse { tot :: [ NAuction ] } deriving (Show,Eq)
\end{verbatim}
onde
\begin{verbatim}
data NAuction = NAuction { a :: Auction, st:: Status } 
                                               deriving (Show,Eq)

data Status = Running | Finished deriving (Eq,Show)
\end{verbatim}
escrever funções de conversão entre os dois formatos,

\begin{itemize}
\item $toNHouse :: House -> NHouse$
\item[-]{Código Haskell:}
\end{itemize}
\begin{verbatim}
toNHouse :: House -> NHouse
toNHouse h = let a = (addd (hrunning h) (Running))
                 b = (addd2 (hfinished h) (Finished))
             in NHouse (a ++ b)

addd :: [Auction] -> Status -> [NAuction]
addd [] _ = []
addd (h:t) a = NAuction (h) (Running) : addd t a

addd2 :: [Auction] -> Status -> [NAuction]
addd2 [] _ = []
addd2 (h:t) a = NAuction (h) (Finished) : addd2 t a
\end{verbatim}
\begin{itemize}
\item[-]{Análise do código:}
\end{itemize}
\paragraph{}O $NHouse$ é uma lista composta por $Auction$, e por $Status$, este que define o estado do item, $Running$ ou $Finished$.
\paragraph{}A função $toNHouse$ primeiro cria duas variáveis, que são as listas $a$ e $b$, a primeira usando a função auxiliar $addd$.
\begin{verbatim}
addd :: [Auction] -> Status -> [NAuction]
addd [] _ = []
addd (h:t) a = NAuction (h) (Running) : addd t a
\end{verbatim}
\paragraph{}Esta função ao receber uma lista de $Auction$ e o estado, cria uma lista de $NAuction$ com cada item da lista $h$ e com o estado, neste caso $Running$.
\paragraph{}A variavel $b$ usa a função auxiliar $addd2$ que que é muito parecida com a $addd$, com a diferença que recebe o estado como $Finished$.
\begin{verbatim}
addd2 :: [Auction] -> Status -> [NAuction]
addd2 [] _ = []
addd2 (h:t) a = NAuction (h) (Finished) : addd2 t a
\end{verbatim}
\paragraph{}Por fim, a função principal ao receber as duas listas, junta-as criando um $NHouse$.
\begin{verbatim}
      ... in NHouse (a ++ b)
\end{verbatim}
%----------------------------------------------------------------------------------------
\begin{itemize}
\item $toHouse :: NHouse -> House$
\item[-]{Código Haskell:}
\end{itemize}
\begin{verbatim}
toHouse :: NHouse -> House
toHouse nh = let r = (map a ((filter ((==Running).st) (tot nh))))
                 f = (map a ((filter ((==Finished).st) (tot nh))))
             in House (r) (f)
\end{verbatim}
\begin{itemize}
\item[-]{Análise do código:}
\end{itemize}
\paragraph{}Esta função tem como objetivo fazer o contrário da anterior. Esta, fazendo uma "triajem" aos estados de cada item facilmente os consegue separar para criar o $House$ pertendido. Para isso função cria duas variáveis $r$ e $f$ a primeira, vai a lista recebida, $nh$, que e um $tot$, e cria outra lista cujos os estados dos itens sejam $Running$, essa lista é depois usada no $map$ que recebe também o $a$ que, segundo a definição do $NAuction$, é o $Auction$.
\begin{verbatim}
      ... r = (map a ((filter ((==Running).st) (tot nh))))
\end{verbatim}
\paragraph{}A segunda variavel usa novamente o $map$ e o $filter$, mas desta vez, aplicado aos itens com o estado $Finished$, criando também uma lista desses mesmos itens. 
\begin{verbatim}
      ... f = (map a ((filter ((==Finished).st) (tot nh))))
\end{verbatim}
\paragraph{}Por ultimo, a função principal, a partir das listas $r$ e $f$ constroi o $House$:
\begin{verbatim}
      ... in House (r) (f)
\end{verbatim}

%xxxxxxxxxxxxxxxxxxxxxxxxxxxxxxxxxxxxxxxxxxxxxxxxxxxxxxxxxxxxxxxxxxxxxxxxxxxxxxxxxxxxxxxxxxxxxxxxxxxxxxxxxxxxxxxxxxxxxxxxxxxxxxxxxxxxxxxxxxxx
\newpage
%xxxxxxxxxxxxxxxxxxxxxxxxxxxxxxxxxxxxxxxxxxxxxxxxxxxxxxxxxxxxxxxxxxxxxxxxxxxxxxxxxxxxxxxxxxxxxxxxxxxxxxxxxxxxxxxxxxxxxxxxxxxxxxxxxxxxxxxxxxxx

\item
Redefinir o tipo $Auction$ acrescentando-lhe mais um campo --- $actclass$, 
\begin{verbatim}
data Auction =  Auction {
                  actid     :: Int,
                  actowner  :: String,
                  actdesc   :: String,
                  actclass  :: String,
                  actvalue  :: Int,
                  actbidder :: String
                } deriving (Show,Eq,Ord) 
\end{verbatim}
que classifica itens em categorias (eg.\ electro-doméstico, mobiliário) e
escrever uma nova função de administração que totaliza os valores leiloados
por categoria ($x$ de electro-domésticos, $y$ de mobiliário, etc).

\begin{itemize}
\item {Tipo de dados:}
\begin{verbatim}
      tClass :: House -> [TotClass]
\end{verbatim}
\end{itemize}
\paragraph{}Recebendo um $House$ devolve uma lista de $TotClass$. Mas o que é um $TotClass$?
\begin{verbatim}
data TotClass = TotClass {classe :: String, valor :: Int} 
                                                deriving Show
\end{verbatim}
\paragraph{}Definimos um $TotClass$ sendo composto por $classe$ e por $valor$, este que totaliza os valores dos itens leiloados de uma determinada $classe$.
\begin{itemize}
\item {Código Haskell:}
\end{itemize}
\begin{verbatim}
tClass :: House -> [TotClass]
tClass h = let l = hrunning h ++ hfinished h	
               q = nub(map actclass (l))
           in lista l q

lista :: [Auction] -> [String]-> [TotClass]
lista l [] = [] 
lista l (h:t) = let a = filter((==h).actclass)l
                    b = map (actvalue) (a)
		in TotClass (h) (foldr (+) 0 b):(lista l t)

data TotClass = TotClass {classe :: String, valor :: Int} 
                                                deriving Show
\end{verbatim}
\begin{itemize}
\item{Análise do código:}
\end{itemize}
\paragraph{}A função principal começa como muitas outras por criar duas variaveis, a primeira $l$, como já tinhamos visto anteriormente cria uma lista com os artigos que estão no $hrunning$ $h$, e os que estão no $hfinished$ $h$. A segunda, $q$, usando o $map$ para que, indo aos $actclass$ dos itens lista $l$ crie uma lista com todas as classes existentes na lista $l$. Em seguida a função $nub$ ao receber essa lista e elimina, caso exista, todas as classes repetidas.
\begin{verbatim}
      ... q = nub(map actclass (l))
\end{verbatim}
\newpage
\paragraph{}De seguida chama a função auxiliar $lista$. Esta função é que faz o trabalho principal, tendo também duas variaveis, a primeira, desta vez $a$, usando o $filter$ cria um lista que recorrendo á lista recebida, $l$, vai pegar os item com a classe igual as classes recebidas na lista.
\begin{verbatim}
      ... a = filter((==h).actclass)l
\end{verbatim}
\paragraph{}A segunda variavel, $b$, usando o $map$, pega na lista $a$ e de cada item vai buscar o seu valor, $actvalue$.
\begin{verbatim}
      ... b = map (actvalue) (a)
\end{verbatim}
\paragraph{}Por ultimo a função $lista$, para construir o $TotClass$, interpreta o $h$ como sendo a classe e usando uma função existente em Haskell, $foldr$\footnote{foldr :: $(a -> b -> b) -> b -> [a] -> b$}, que recebendo a instrução de soma, o caso de paragem, neste caso o elemento neutro da soma, e a lista vinda de $b$, soma todos os valores dos artigos com classes que estejam na lista recebida, calculando com o uso a recursivamente. Formando assim um $TotClass$ para cada classe.
\begin{verbatim}
       ... in TotClass (h) (foldr (+) 0 b):(lista l t)

\end{verbatim}

%xxxxxxxxxxxxxxxxxxxxxxxxxxxxxxxxxxxxxxxxxxxxxxxxxxxxxxxxxxxxxxxxxxxxxxxxxxxxxxxxxxxxxxxxxxxxxxxxxxxxxxxxxxxxxxxxxxxxxxxxxxxxxxxxxxxxxxxxxxxx
%xxxxxxxxxxxxxxxxxxxxxxxxxxxxxxxxxxxxxxxxxxxxxxxxxxxxxxxxxxxxxxxxxxxxxxxxxxxxxxxxxxxxxxxxxxxxxxxxxxxxxxxxxxxxxxxxxxxxxxxxxxxxxxxxxxxxxxxxxxxx

\item	Um dos defeitos do modelo de dados $Auction$ é não guardar o valor
        inicial de licitação. Corrijir esta limitação, criando ainda uma
        função que contabilize quanto é que, em média, os itens vendidos
        se valorizaram no leilão.
\begin{itemize}
\item {Tipo de dados:}
\begin{verbatim}
      media :: House -> (Float, String)
\end{verbatim}
\end{itemize}
\paragraph{}Para esta função acrescentamos ao tipo $Auction$ outro campo "actvalinc", que guarda o valor inicial. O resultado desta função é, como diz no timo, $(Float, String)$, sendo o $(Float)$ a diferença média dos artigos, e consuante esse valor, exprime na $String$ de valurizou ou não valurizou.
\begin{itemize}
\item{Código Halkell:}
\end{itemize}
\begin{verbatim}
media :: House -> (Float, String)
media h = let l = hfinished h
              a = map actvalinc l
              b = map actvalue l
          in auxMedia (zip a b)

auxMedia :: (Fractional t, Integral a, Ord t)=>[(a, a)]->(t, String)
auxMedia x = let a = map (\(m,n) -> n-m)x
                 b = fromIntegral (sum a)
                 c = fromIntegral (length a)
                 d = b/c
             in if (d>0) then (d,"Valorizou") 
                              else (d,"Nao valorizou") 
\end{verbatim}
\newpage
\begin{itemize}
\item{Análise do Código:}
\end{itemize}
\paragraph{}A função principal, cria três listas como variáveis, $l$ que como os itens vendidos, a lista $l$ só utiliza o $hfinished$ $h$. A segunda, $a$, e a treceira ,$b$, são semelhantes, mas a primeira cria uma lista dos valores iniciais, $actvalinc$, enquanto a segunda cria dos valores a que os produtos foram vendidos, $actvalue$. Chando de seguida a função auxiliar $auxMedia$, que atravez da função $zip$\footnote{$zip :: [a] -> [b] -> [(a,b)]$} recebe uma lista de pares, com o valor inicial, e o valor final.  
\paragraph{}A função auxiliar,
\begin{verbatim}
auxMedia :: (Fractional t, Integral a, Ord t)=>[(a, a)]->(t, String)
auxMedia x = let a = map (\(m,n) -> n-m)x
                 b = fromIntegral (sum a)
                 c = fromIntegral (length a)
                 d = b/c
             in if (d>0) then (d,"Valorizou") 
                              else (d,"Nao valorizou") 
\end{verbatim}
tem como objetivo tratar das operações. Esta atravéz do $let$ faz grande parte do trabalho, a variável $a$ resulta de aplicar o $map$ com uma função anónima, que pegando na lista que a função recebe como argumento, faz a subtração do valor final pelo valor inicial. Nas duas seguintes, $b$ e $c$, a primeira faz o sumatório dos valores resultantes de $a$, e a segunda calcula o comprimento da lista $a$. A utilização do $fromIntegral$ é para mais tarde poder utilizar a barra de divisão. A última, $d$, apenas faz a operação de subtração.
\paragraph{}Por último a função faz uma pequena verificação, se o valor da média for maior que zero, significa que o produto valorizou, caso contrário não valorizou.
\begin{verbatim}
   ... in if (d>0) then (d,"Valorizou") else (d,"Nao valorizou")
\end{verbatim} 
\end{enumerate}
\end{document}
